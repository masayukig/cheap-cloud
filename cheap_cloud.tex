\documentclass[aspectratio=169,11pt,hyperref={colorlinks=true}]{beamer}
% https://github.com/zr-tex8r/BXcjkjatype/blob/master/README-ja.md
\usepackage[whole]{bxcjkjatype}
\usetheme{boxes}
\setbeamertemplate{navigation symbols}{}
\definecolor{suse}{RGB}{2, 211, 95}
\definecolor{susedark}{RGB}{13, 44, 64}
\setbeamercolor{titlelike}{fg=suse}
\setbeamercolor{structure}{fg=suse}
\hypersetup{colorlinks,urlcolor=suse}
\setbeamertemplate{footline}[frame number]
% Inserting graphics
\usepackage{graphicx}
% Side-by-side figures, etc
\usepackage{subfigure}
% Code snippits
\usepackage{listings}
% Color stuff
\usepackage{color}
% underline
\usepackage{soul}
% calc
\usepackage{calc}

\usepackage{amsmath}
\usepackage{tikz}
\newcommand\RBox[1]{%
  \tikz\node[draw,rounded corners,align=center,] {#1};%
}
\usepackage{hyperref}
%\usecolortheme{buzz}
%\usecolortheme{wolverine}
%\usetheme{Boadilla}
\usepackage[T1]{fontenc}
%\usepackage{fontspec}
%\usepackage[expert, deluxe]{otf}

\definecolor{mygreen}{rgb}{0,0.6,0}
\definecolor{mygray}{rgb}{0.5,0.5,0.5}
\definecolor{mymauve}{rgb}{0.58,0,0.82}


%\usepackage{CJK}
%\pdfmapline{=genshingothic@Unicode@ <genshingothic.ttf}
% bxcjkjatype
%\setgothicfont[<ID>]{<フォントファイル名>}
%\setgothicfont{/Users/foo/Library/Fonts/genshingothic.ttf}
%\setgothicfont{/Users/foo/Library/Fonts/NotoSansCJKjp-Regular.otf}
%\setgothicfont{/Users/foo/Downloads/genshingothic-20150607/GenShinGothic-P-Normal.ttf}
%\setgothicfont{/Users/foo/Downloads/genshingothic-20150607/GenShinGothic-Regular.ttf}
%\setgothicfont{hiragino.ttc}
\setgothicfont{mplus-1p-regular.ttf}
\setCJKfamilydefault{\gtdefault}
%\setCJKfamilydefault{\mcdefault}
%\CJKforce{abcdefghijklmnopqrstuvwxyzABCDEFGHIJKLMNOPQRSTUVWXYZ}


\lstset{%
  backgroundcolor=\color{susedark},   % choose the background color; you must add \usepackage{color} or \usepackage{xcolor}
  breakatwhitespace=false,         % sets if automatic breaks should only happen at whitespace
  breaklines=true,                 % sets automatic line breaking
  captionpos=b,                    % sets the caption-position to bottom
  commentstyle=\color{suse},  % comment style
  extendedchars=true,              % lets you use non-ASCII characters; for 8-bits encodings only, does not work with UTF-8
  keepspaces=true,                 % keeps spaces in text, useful for keeping indentation of code (possibly needs columns=flexible)
  keywordstyle=\color{blue},       % keyword style
%  otherkeywords={*,...},           % if you want to add more keywords to the set
  numbersep=5pt,                   % how far the line-numbers are from the code
  numberstyle=\tiny\color{mygray}, % the style that is used for the line-numbers
  rulecolor=\color{white},         % if not set, the frame-color may be changed on line-breaks within not-black text (e.g. comments (green here))
  showspaces=false,                % show spaces everywhere adding particular underscores; it overrides 'showstringspaces'
  showstringspaces=false,          % underline spaces within strings only
  showtabs=false,                  % show tabs within strings adding particular underscores
  stringstyle=\color{suse},   % string literal style
}

\setbeamerfont{caption}{series=\normalfont,size=\fontsize{6}{8}}
%\setbeamerfont{caption}{series=\normalfont,size=\large}
\setbeamertemplate{caption}{\raggedright\insertcaption\par}

\setlength{\abovecaptionskip}{0pt}
\setlength{\floatsep}{0pt}

\author[Masayuki Igawa]{%
    \texorpdfstring{%
        \begin{columns}
        \column{.45\linewidth}
            \centering
            Masayuki Igawa\\
            \href{mailto:masayuki@igawa.io}{masayuki@igawa.io}\\
            \texttt{masayukig on Freenode, GitHub, Twitter}
        \end{columns}
        }
    {Masayuki Igawa}
}
\date{June 23, 2018}
\def\place#1{\def\@place{#1}}
\place{\href{https://opensuseja.connpass.com/event/86085/}{@openSUSE mini Summit 2018}}

\title[private-cloud-on-openSUSE
  \hspace{2em}\insertframenumber/\inserttotalframenumber]{openSUSE で おうちクラウド
  \\ @openSUSE mini Summit 2018}

\setbeamercolor{background canvas}{bg=susedark}
\setbeamercolor{titlelike}{fg=white}
\setbeamercolor{structure}{fg=white}
\setbeamercolor{normal text}{fg=white}

\begin{document}

{%
% \setbeamertemplate{background canvas}{\includegraphics[width=\paperwidth,height=\paperheight]{background_title.png}}
\setbeamertemplate{footline}{}
\setbeamercolor{background canvas}{bg=susedark}
\begin{frame}[noframenumbering]
  \hypersetup{colorlinks,urlcolor=suse}
  \setbeamercolor{author}{fg=white}
  \setbeamercolor{date}{fg=white}
  \setbeamercolor{place}{fg=white}
  \titlepage{}
  \centering
  \@place \par
  \href{https://github.com/masayukig/cheap-cloud}{https://github.com/masayukig/cheap-cloud}
  \vspace{1em}
  \begin{flushright}
    \tiny\href{https://creativecommons.org/licenses/by/4.0/}{This work
      is licensed under a Creative Commons Attribution 4.0
      International License.}~\includegraphics[scale=0.3]{cc_by.png}
  \end{flushright}
\end{frame}
}

\section{Agenda}
\begin{frame}
  \frametitle{Agenda}
  \begin{itemize}
    \item Who I am?
    \item What is ``the OpenStack''?
    \item What I did
    \item Why do I need it?
    \item How to build it?
    \item Benefits
    \item Issues
    \item Conclusion
    \item Demo
  \end{itemize}
\end{frame}

\section{Who I am?}
\begin{frame}
  \frametitle{Who I am?}
  \begin{itemize}
    \item Company:SUSE/ノベル株式会社
      \begin{itemize}
        \item SUSE OpenStack Cloud QE(Quality Engineering) Team (日本にいるのは私だけ)
        \item[] \href{https://www.suse.com/newsroom/post/2016/suse-acquires-openstack-iaas-and-cloud-foundry-paas-talent-and-technology-assets-from-hpe-to-accelerate-growth-and-entry-into-new-markets/}{SUSE Acquires OpenStack IaaS and Cloud Foundry PaaS Talent and Technology Assets from HPE to Accelerate Growth and Entry into New Markets}
      \end{itemize}
    \item Job: Senior Software Engineer/Open Source Programmer
      \begin{itemize}
        \item \href{https://www.openstack.org/}{OpenStack}
         \href{https://wiki.openstack.org/wiki/QA}{QA} Upstream/Downstream development, Core Reviewer
        \item[] (\href{https://docs.openstack.org/developer/tempest/}{Tempest},
         \href{http://status.openstack.org/openstack-health/}{OpenStack-Health},
         \href{https://docs.openstack.org/developer/subunit2sql/}{Subunit2SQL},
         \href{https://docs.openstack.org/developer/stackviz/}{Stackviz})
        \item \href{http://stackalytics.com/?user_id=igawa&release=all&metric=all}{stackalytics.com/?user\_id=igawa}
        \item \href{https://github.com/masayukig}{github.com/masayukig}
      \end{itemize}
    \item Books \includegraphics[scale=0.2]{OpenStack_Integration_book.png}~\includegraphics[scale=0.2]{InfraCI_book.png}
      \begin{itemize}
      \item \href{https://www.amazon.co.jp/dp/4798139785/}{OpenStack
        クラウドインテグレーション (オープンソースクラウドによるサービ
        ス構築入門)}
      \item \href{https://www.amazon.co.jp/dp/4798155128/}{インフラ CI
        実践ガイド (Ansible/GitLabを使ったインフラ改善サイクルの実
        現)} (レビュー参加)
      \end{itemize}
  \end{itemize}
\end{frame}

\section{What is ``the OpenStack''?}
\begin{frame}
  \frametitle{What is ``the OpenStack''?}
  \begin{itemize}
    \item Open Source Cloud OS Software: Apache License Version 2.0
    \item Written in Python
    \item There are a lot of `OpenStack' projects: \href{http://governance.openstack.org/reference/projects/index.html}{65 projects(2018-06-18)}
    \item Released every 6 month: Latest version is called `Queens'
    \item Users: AT\&T, American Airlines, BBVA, Bloomberg, CERN,
      China, Comcast, Gap, Mobile, Nike, VEXXHOST, Verizon,
      Volkswagen, WALMART, eBay, etc..
  \end{itemize}
\end{frame}

\section{What I did}
\begin{frame}
  \frametitle{What I did}
  \begin{itemize}
    \item Got 1U servers * 3
    \item Set up the servers
    \item Installed openSUSE
    \item Installed OpenStack
    \item Using VMs
  \end{itemize}
\end{frame}

\section{Why do I need the private Cloud?}
\begin{frame}
  \frametitle{Why do I need the private Cloud?}
  \begin{itemize}
    \item Understand architecture of OpenStack
    \item Use VMs for sandboxes such as k8s, mesos, etc.
    \item \bf{FUN!!}
  \end{itemize}
\end{frame}

\section{How to build}
\begin{frame}
  \frametitle{Yahoo! Auction}
  \begin{columns}[T]
    \begin{column}{0.45\textwidth}
      \begin{figure}
        \begin{center}
          \includegraphics[width=1.0\textwidth]{yahoo_auction_1u.png}
        \end{center}
      \end{figure}
    \end{column}
    \begin{column}{0.45\textwidth}
      \begin{figure}
        \begin{center}
          \includegraphics[width=1.0\textwidth]{yahoo_auction_1u_large.png}
        \end{center}
      \end{figure}
    \end{column}
  \end{columns}
\end{frame}

\begin{frame}
  \frametitle{Get 1U servers * 3}
  \begin{itemize}
    \item Yahoo! Auction!! \\
      Dell PowerEdge R410 * 3: 59.58k JPY\\
      \includegraphics[scale=1.0]{auction_r410.png}\\
    \item[] 1U, Xeon L5640(6cores * 2CPU HT), 32GB RAM, 250GB HDD*2
    \item[] (Cost: 18.52k JPY * 3 servers = 55.56k, 4.02k (for a rental car))
  \end{itemize}
\end{frame}

\begin{frame}
  \frametitle{Install the servers}
  \begin{itemize}
    \item Stack on the floor? -> Hard to move
    \item Rack? -> Too expensive
    \item \href{https://wiki.eth0.nl/index.php/LackRack}{LackRack}
    \item [TODO] Add Pictures for LackRack
  \end{itemize}
\end{frame}

\begin{frame}
  \frametitle{Install openSUSE}
  No automation such as autoyast, ansible, puppet, etc
  \begin{itemize}
    \item Download image and burn it to a USB stick (\url{https://software.opensuse.org/distributions/leap})
    \item Install from that media
    \item Update it to the latest: \texttt{\$ sudo zypper dup}
    \item [TODO] Add Pictures for openSUSE
  \end{itemize}
\end{frame}

\begin{frame}
  \frametitle{Install OpenStack}
  Use openSUSE rpm packages
  \begin{itemize}
    \item Read the Doc (e.g. \url{https://docs.openstack.org/nova/latest/})
    \item Install from the openSUSE repo
    \item Configure
    \item [TODO] Add Pictures for OpenStack dashboard
  \end{itemize}
\end{frame}

\begin{frame}
  \frametitle{Update OpenStack}
  \begin{itemize}
    \item Install from the openSUSE repo (Just needed to update repo URL)
    \item No Configuration changes
  \end{itemize}
\end{frame}

\section{Benefits}
\begin{frame}
  \frametitle{Benefits}
  \begin{itemize}
    \item Free to use!!!
    \item Low Cost to start
    \item Powerful
    \item Low Network Latency
    \item Warm (in winter)
  \end{itemize}
\end{frame}

\section{Issues}
\begin{frame}
  \frametitle{Issues}
  \begin{itemize}
    \item Electricity cost: 10,000 JPY/month (Expensive)
    \item Noise (Imagin a server room)
    \item Space (Expensive in Tokyo)
    \item Failures (HDD, Power Unit... Expensive)
    \item Abandonment (Expensive)
  \end{itemize}
\end{frame}

\section{Demo}
\begin{frame}
  \frametitle{Demo}
  \begin{itemize}
    \item TBD: Boot an Instance or Cloud Native something?
    \item [TODO] Add images for demo just in case(?)
  \end{itemize}
\end{frame}

\section{Future work}
\begin{frame}
  \frametitle{Future work}
  \begin{itemize}
    \item Replace the broken HDD to SSD
    \item Upgrade openSUSE to Leap 15.0
    \item Use a RaspberryPi as a controller node
  \end{itemize}
\end{frame}

\section{Conclusion}
\begin{frame}
  \frametitle{Conclusion}
  \begin{itemize}
    \item Initial cost could be low but \bf{EXPENSIVE} to maintain and \bf{NOSIY}
    \item More transparent than public clouds
    \item Own physical servers and play with it is super \bf{FUN}!
  \end{itemize}
\end{frame}

\end{document}
